\documentclass{article}
\usepackage{avitaawesome}
\usepackage{mathrsfs}
\usepackage[retainorgcmds]{IEEEtrantools}
\usepackage[margin=1in]{geometry}
\usepackage{inconsolata}
\usepackage{graphicx}
\usepackage{color}
\usepackage{fancyvrb}
\usepackage[dvipsnames]{xcolor}
\usepackage{ragged2e}

%\DeclareMathSizes{10}{10}{10}{10}
%%%%%%%%%%%%%%%%%%%%%%%%%%%%%%%%%%%%%%%%%%%%%%%%%%%%%%
%%%%% End setting up basic program parameters
%%%%%%%%%%%%%%%%%%%%%%%%%%%%%%%%%%%%%%%%%%%%%%%%%%%%%%

%%%%%%%%%%%%%%%%%%%%%%%%%%%%%%%%%%%%%%%%%%%%%%%%%%%%%%
%%%%% Begin setting up theorems, definitions, etc.
%%%%%%%%%%%%%%%%%%%%%%%%%%%%%%%%%%%%%%%%%%%%%%%%%%%%%%

\newtheorem{theorem}{Theorem}[section]
\newtheorem{conjecture}[theorem]{Conjecture}
\newtheorem{corollary}[theorem]{Corollary}
\newtheorem{lemma}[theorem]{Lemma}
\newtheorem{proposition}[theorem]{Proposition}
\newtheorem{remark}[theorem]{Remark}

\theoremstyle{definition}

\newtheorem{definition}[theorem]{Definition}
\newtheorem{example}[theorem]{Example}
\newtheorem{problem}[theorem]{Problem}

%%%%%%%%%%%%%%%%%%%%%%%%%%%%%%%%%%%%%%%%%%%%%%%%%%%%%%
%%%%% End setting up theorems, definitions, etc.
%%%%%%%%%%%%%%%%%%%%%%%%%%%%%%%%%%%%%%%%%%%%%%%%%%%%%%

\author{Avita Sharma, Eric Wyss, and David Taus}
\title{\textbf{Project Proposal}}
\date{}

\begin{document}
\maketitle
\section*{Team Name: DragonSlayers}
\section*{Current Group Members:}
 Avita Sharma, Eric Wyss, and David Taus
\section*{Project: DM or Die}
A text based adventure game, where the user controls the dungeon master. They can 
manipulate intelligent `player characters' who interact in the game. Example: the
user can decide if a player succeeds or fails at a task. These players are 
concurrent processes which make demands on the dungeon master. These demands
must be answered in real time or a default behavior is triggered, the default
generally having a negative consequence. Multiple demands can come from each 
`player character' at the same time, and the user must decide which one to interact 
with (or if they are quick, try to do them all). The user can directly control
monsters and non-playable characters (NPCs) for the `player characters' to interact 
with.
\section*{Minimum deliverable:}
\begin{enumerate}
\item [*] Split Screen graphical interface.
\item [*] Two `Player Characters' with a set alignment and class.
\item [*] One Encounter involving talking to one NPC and one battle. 
\item [*] The default map has a dungeon and tavern.
\item [*] Ability to pause the game.
\item [*] Default/unchangeable inventory.
\end{enumerate}
\section*{Maximum deliverable:}
\begin{enumerate}
\item [*] Implement up to six characters.
\item [*] More classes, more monsters.
\item [*] Ability to add more places and characters to the map.
\item [*] Add more encounters. Increase the complexity of encounters.
\item [*] Basic Excitement meter for each `player character'. The game is scored
		  on total excitement. Excitement is
		  a metric for how much fun the `player characters' are having with the
		  user's decisions. (i.e. not purposely failing the battle.) Excitement $\in \RR$.
\item [*] More user customizable options for the campaign.
\item [*] Option to save the game.
\end{enumerate}
\section*{First Step:}
\begin{enumerate}
\item [*] One `player character'. (One alignment, one class.)
\item [*] Message handling between user and `player character'.
\item [*] One NPC character for the user to control.
\item [*] One combat encounter; one monster for the user to control.
\item [*] End of combat ends the game with, `See you next session!'
\item [*] Combat ends if the `player character' dies or the user's
		  monster (dragon) dies. 
\end{enumerate}
\section*{Foreseeable Problems:}
\begin{enumerate}
\item [*] Good design documents.
\item [*] `Player characters' spam messages too fast for the user to keep up.
\item [*] Too slow response times.
\item [*] Not enough play-testers.
\item [*] Too difficult.
\item [*] Handling contradictory events.
\end{enumerate}
\end{document}
